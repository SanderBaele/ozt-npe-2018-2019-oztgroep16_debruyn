%==============================================================================
% Voorbeeld gebruik documentklasse hogent-article
%==============================================================================
%
% Compileren in TeXstudio:
%
% - Zorg dat Biber de bibliografie compileert (en niet Biblatex)
%   Options > Configure > Build > Default Bibliography Tool: "txs:///biber"
% - F5 om te compileren en het resultaat te bekijken.
% - Als de bibliografie niet zichtbaar is, probeer dan F5 - F8 - F5
%   Met F8 compileer je de bibliografie apart.
%
% Als je JabRef gebruikt voor het bijhouden van de bibliografie, zorg dan
% dat je in ``biblatex''-modus opslaat: File > Switch to BibLaTeX mode.

\documentclass{hogent-article}

\usepackage{lipsum} % Voor vultekst

%------------------------------------------------------------------------------
% Metadata over het artikel
%------------------------------------------------------------------------------

%---------- Titel & auteur ----------------------------------------------------

% TODO: geef werktitel van je eigen voorstel op
\PaperTitle{Titel van het artikel}
% TODO: geef op welk soort artikel dit is
% Dit is typisch de opdracht en het vak waarvoor dit artikel geschreven is, bv.
% ``Verslag onderzoeksproject Onderzoekstechnieken 2018-2019''
\PaperType{Type artikel}

% TODO: vul je eigen naam in als auteur, geef ook je emailadres mee!
\Authors{Robbe Decorte\textsuperscript{1}, Jan Janssens\textsuperscript{2}} % Authors

% TODO: vul de naam van je co-promotor in.
% Als het hier gaat om een voorstel voor de bachelorproef, dan ben je hier
% verplicht de naam van je co-promotor in te vullen. Zoniet, dan kan je het
% leeg laten.
\CoPromotor{}

% Contactinfo: Geef hier de contactgegevens van elke auteur van het artikel (en
% indien van toepassing ook van de co-promotor).
\affiliation{
	\textsuperscript{1} \href{mailto:robbe.decorte@student.hogent.be}{robbe.decorte@student.hogent.be}}
\affiliation{
	\textsuperscript{2} \href{mailto:jan.janssens.u4321@student.hogent.be}{mailto:jan.janssens.u4321@student.hogent.be}
}

%---------- Abstract ----------------------------------------------------------

\Abstract{Hier schrijf je de samenvatting van je artikel, als een doorlopende tekst van één paragraaf. Wat hier zeker in moet vermeld worden: \textbf{Context} (Waarom is dit werk belangrijk?); \textbf{Nood} (Waarom moet dit onderzocht worden?); \textbf{Taak} (Wat ga je (ongeveer) doen?); \textbf{Object} (Wat staat in dit document geschreven?); \textbf{Resultaat} (Wat verwacht je van je onderzoek?); \textbf{Conclusie} (Wat verwacht je van van de conclusies?); \textbf{Perspectief} (Wat zegt de toekomst voor dit werk?).
	
	Bij de sleutelwoorden geef je het onderzoeksdomein, samen met andere sleutelwoorden die je werk beschrijven.
}

%---------- Onderzoeksdomein en sleutelwoorden --------------------------------
% TODO: Vul de sleutelwoorden aan.


\Keywords{Onderzoeksdomein; Learning; Retention; Leertechnieken}
\newcommand{\keywordname}{Sleutelwoorden} % Defines the keywords heading name

%---------- Titel, inhoud -----------------------------------------------------

\begin{document}
	
	\flushbottom % Makes all text pages the same height
	\maketitle % Print the title and abstract box
	\tableofcontents % Print the contents section
	\thispagestyle{empty} % Removes page numbering from the first page
	
	%------------------------------------------------------------------------------
	% Hoofdtekst
	%------------------------------------------------------------------------------
	
	\section{Inleiding}
	
	
	
	\section{Overzicht literatuur}
	
	% Refereren naar de literatuur kan met:
	% \autocite{BIBTEXKEY} -> (Auteur, jaartal)
	% \textcite{BIBTEXKEY} -> Auteur (jaartal)
	Evaluating storage, retention, and retrieval in disordered memory and learning~\autocite{BuschkeFuld1974}
	
	Ze bespreken 2 simpele manieren om bij verschillende mensen het geheugen te quoteren en met
	elkaar te vergelijken (nl. selectief herinneren en restrictief herinneren)
	Bij selectief herinneren herhalen ze enkel de woorden die je niet genoemd hebt in de test die direct volgt na het
	opzeggen van alle woorden tot je 2 opeenvolgende testen alle woorden kan opzeggen, restrictief gaat doortot je elk woord minstens
	1x kan herinneren zodat de lijst inkrimpt (elk woord wordt na het herinneren ervan uit de lijst gehaald).
	Woorden die je kan herinneren in een sessie waar die voorgaand niet in gezegd is, komen uit je lange
	termijn geheugen. Het falen van de patiënt kunnen we linken aan het niet (snel genoeg) kunnen onthouden van woorden in het
	lange termijn geheugen. We kunnen zeggen dat een lijst gekend is als de patiënt consistent alle woorden kan herhalen, het heeft dus geen nut als
	je uit het lange termijn geheugen woorden kan onthouden wanneer je andere woorden die net gezegd zijn niet kan herhalen.
	Deze methoden zijn enkel nuttig om abnormaliteiten vast te stellen (in dit geval alcoholisme) maar vertellen weinig over de staat van het geheugen
	als je 2 patiënten met hetzelfde probleem met elkaar vergelijkt. De patiënt had bij beide methoden 6 pogingen nodig vooraleer die in staat was om alle woorden
	op te zeggen, maar was niet in staat om deze te herhalen in de volgende sessie van selectief herinneren. De beste manier om dit aan te pakken is door je visie te 
	veranderen, je mag het niet zien als 10 individuele woorden maar als een lijst waar je zelf een link moet leggen tussen verschillende woorden. Op deze manier kunnen wij
	een hypothese opstellen, "Door een link te leggen tussen de woorden zorg je niet alleen dat je in de volgende sessie een hoger aantal woorden kan herhalen maar ook 
	dat je na enkele uren meer woorden kan opsommen dan iemand die deze manier niet toepast". Deze hypothese lijkt me belangrijk voor het resultaat maar is niet behandeld in deze studie.
	
	
	\section{Methodologie}
	

	
	\section{Experimenten}
	
	
	
	\section{Analyse resultaten}
	

	
	\section{Conclusie}
	
	
	
	%------------------------------------------------------------------------------
	% Referentielijst
	%------------------------------------------------------------------------------
	% TODO: de gerefereerde werken moeten in BibTeX-bestand ``bibliografie.bib''
	% voorkomen. Gebruik JabRef om je bibliografie bij te houden en vergeet niet
	% om compatibiliteit met Biber/BibLaTeX aan te zetten (File > Switch to
	% BibLaTeX mode)
	
	\phantomsection
	\printbibliography[heading=bibintoc]
	
\end{document}
