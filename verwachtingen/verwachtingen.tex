%==============================================================================
% Voorbeeld gebruik documentklasse hogent-article
%==============================================================================
%
% Compileren in TeXstudio:
%
% - Zorg dat Biber de bibliografie compileert (en niet Biblatex)
%   Options > Configure > Build > Default Bibliography Tool: "txs:///biber"
% - F5 om te compileren en het resultaat te bekijken.
% - Als de bibliografie niet zichtbaar is, probeer dan F5 - F8 - F5
%   Met F8 compileer je de bibliografie apart.
%
% Als je JabRef gebruikt voor het bijhouden van de bibliografie, zorg dan
% dat je in ``biblatex''-modus opslaat: File > Switch to BibLaTeX mode.

\documentclass{hogent-article}

\usepackage{lipsum} % Voor vultekst

%------------------------------------------------------------------------------
% Metadata over het artikel
%------------------------------------------------------------------------------

%---------- Titel & auteur ----------------------------------------------------

% TODO: geef werktitel van je eigen voorstel op
\PaperTitle{Titel van het artikel}
% TODO: geef op welk soort artikel dit is
% Dit is typisch de opdracht en het vak waarvoor dit artikel geschreven is, bv.
% ``Verslag onderzoeksproject Onderzoekstechnieken 2018-2019''
\PaperType{Type artikel}

% TODO: vul je eigen naam in als auteur, geef ook je emailadres mee!
\Authors{Steven Stevens\textsuperscript{1}, Jan Janssens\textsuperscript{2}} % Authors

% TODO: vul de naam van je co-promotor in.
% Als het hier gaat om een voorstel voor de bachelorproef, dan ben je hier
% verplicht de naam van je co-promotor in te vullen. Zoniet, dan kan je het
% leeg laten.
\CoPromotor{}

% Contactinfo: Geef hier de contactgegevens van elke auteur van het artikel (en
% indien van toepassing ook van de co-promotor).
\affiliation{
	\textsuperscript{1} \href{mailto:steven.stevens.u1234@student.hogent.be}{steven.stevens.u1234@student.hogent.be}}
\affiliation{
	\textsuperscript{2} \href{mailto:jan.janssens.u4321@student.hogent.be}{mailto:jan.janssens.u4321@student.hogent.be}
}

%---------- Abstract ----------------------------------------------------------

\Abstract{Hier schrijf je de samenvatting van je artikel, als een doorlopende tekst van één paragraaf. Wat hier zeker in moet vermeld worden: \textbf{Context} (Waarom is dit werk belangrijk?); \textbf{Nood} (Waarom moet dit onderzocht worden?); \textbf{Taak} (Wat ga je (ongeveer) doen?); \textbf{Object} (Wat staat in dit document geschreven?); \textbf{Resultaat} (Wat verwacht je van je onderzoek?); \textbf{Conclusie} (Wat verwacht je van van de conclusies?); \textbf{Perspectief} (Wat zegt de toekomst voor dit werk?).
	
	Bij de sleutelwoorden geef je het onderzoeksdomein, samen met andere sleutelwoorden die je werk beschrijven.
}

%---------- Onderzoeksdomein en sleutelwoorden --------------------------------
% TODO: Vul de sleutelwoorden aan.


\Keywords{Onderzoeksdomein; Sleutelwoord1; Sleutelwoord2; Sleutelwoord3}
\newcommand{\keywordname}{Sleutelwoorden} % Defines the keywords heading name

%---------- Titel, inhoud -----------------------------------------------------

\begin{document}
	
	\flushbottom % Makes all text pages the same height
	\maketitle % Print the title and abstract box
	\tableofcontents % Print the contents section
	\thispagestyle{empty} % Removes page numbering from the first page
	
	%------------------------------------------------------------------------------
	% Hoofdtekst
	%------------------------------------------------------------------------------
	
	\section{Inleiding}
	
	\lipsum[1-3]
	
	\section{Overzicht literatuur}
	
	% Refereren naar de literatuur kan met:
	% \autocite{BIBTEXKEY} -> (Auteur, jaartal)
	% \textcite{BIBTEXKEY} -> Auteur (jaartal)
	Voorbeeld van een referentie~\autocite{Moore2002}
	
	\lipsum[4-9]
	
	\section{Methodologie}
	
	\lipsum[10-12]
	
	\section{Experimenten}
	
	\lipsum[14-18]
	
	\section{Analyse resultaten}
	
	\lipsum[18-21]
	
	\section{Conclusie}
	
	\lipsum[22-23]
	
	%------------------------------------------------------------------------------
	% Referentielijst
	%------------------------------------------------------------------------------
	% TODO: de gerefereerde werken moeten in BibTeX-bestand ``bibliografie.bib''
	% voorkomen. Gebruik JabRef om je bibliografie bij te houden en vergeet niet
	% om compatibiliteit met Biber/BibLaTeX aan te zetten (File > Switch to
	% BibLaTeX mode)
	
	\phantomsection
	\printbibliography[heading=bibintoc]
	
\end{document}
