%==============================================================================
% Sjabloon onderzoeksvoorstel bachelorproef
%==============================================================================
% Gebaseerd op LaTeX-sjabloon ‘Stylish Article’ (zie voorstel.cls)
% Auteur: Jens Buysse, Bert Van Vreckem
%
% Compileren in TeXstudio:
%
% - Zorg dat Biber de bibliografie compileert (en niet Biblatex)
%   Options > Configure > Build > Default Bibliography Tool: "txs:///biber"
% - F5 om te compileren en het resultaat te bekijken.
% - Als de bibliografie niet zichtbaar is, probeer dan F5 - F8 - F5
%   Met F8 compileer je de bibliografie apart.
%
% Als je JabRef gebruikt voor het bijhouden van de bibliografie, zorg dan
% dat je in ``biblatex''-modus opslaat: File > Switch to BibLaTeX mode.

\documentclass{voorstel}

%------------------------------------------------------------------------------
% Metadata over het voorstel
%------------------------------------------------------------------------------

%---------- Titel & auteur ----------------------------------------------------

% TODO: geef werktitel van je eigen voorstel op
\PaperTitle{Titel voorstel}
\PaperType{Onderzoeksvoorstel Onderzoekstechnieken 2018-2019} % Type document

% TODO: vul je eigen naam in als auteur, geef ook je emailadres mee!
\Authors {Robbe Decorte\textsuperscript{1}} % Authors
\CoPromotor{Piet Pieters\textsuperscript{2} (Bedrijfsnaam)}
\affiliation{\textbf{Contact:}
	\textsuperscript{1} \href{mailto:robbe.decorte@student.hogent.be}{steven.stevens.u1234@student.hogent.be};
	\textsuperscript{2} \href{mailto:piet.pieters@acme.be}{piet.pieters@acme.be};
}

%---------- Abstract ----------------------------------------------------------

\Abstract{Hier schrijf je de samenvatting van je voorstel, als een doorlopende tekst van één paragraaf. Wat hier zeker in moet vermeld worden: \textbf{Context} (Waarom is dit werk belangrijk?); \textbf{Nood} (Waarom moet dit onderzocht worden?); \textbf{Taak} (Wat ga je (ongeveer) doen?); \textbf{Object} (Wat staat in dit document geschreven?); \textbf{Resultaat} (Wat verwacht je van je onderzoek?); \textbf{Conclusie} (Wat verwacht je van van de conclusies?); \textbf{Perspectief} (Wat zegt de toekomst voor dit werk?).
	
	Bij de sleutelwoorden geef je het onderzoeksdomein, samen met andere sleutelwoorden die je werk beschrijven.
	
	Vergeet ook niet je co-promotor op te geven.
}

%---------- Onderzoeksdomein en sleutelwoorden --------------------------------
% TODO: Sleutelwoorden:
%
% Het eerste sleutelwoord beschrijft het onderzoeksdomein. Je kan kiezen uit
% deze lijst:
%
% - Mobiele applicatieontwikkeling
% - Webapplicatieontwikkeling
% - Applicatieontwikkeling (andere)
% - Systeembeheer
% - Netwerkbeheer
% - Mainframe
% - E-business
% - Databanken en big data
% - Machineleertechnieken en kunstmatige intelligentie
% - Andere (specifieer)
%
% De andere sleutelwoorden zijn vrij te kiezen

\Keywords{Onderzoeksdomein. Onderzoekstechnieken --- Learning --- Retention} % Keywords
\newcommand{\keywordname}{Sleutelwoorden} % Defines the keywords heading name

%---------- Titel, inhoud -----------------------------------------------------

\begin{document}
	
	\flushbottom % Makes all text pages the same height
	\maketitle % Print the title and abstract box
	\tableofcontents % Print the contents section
	\thispagestyle{empty} % Removes page numbering from the first page
	
	%------------------------------------------------------------------------------
	% Hoofdtekst
	%------------------------------------------------------------------------------
	
	% De hoofdtekst van het voorstel zit in een apart bestand, zodat het makkelijk
	% kan opgenomen worden in de bijlagen van de bachelorproef zelf.
	%---------- Inleiding ---------------------------------------------------------
	
	\section{Introductie} % The \section*{} command stops section numbering
	\label{sec:introductie}
	
	Hier introduceer je werk. Je hoeft hier nog niet te technisch te gaan.
	
	Je beschrijft zeker:
	
	\begin{itemize}
		\item de probleemstelling en context
		\item de motivatie en relevantie voor het onderzoek
		\item de doelstelling en onderzoeksvraag/-vragen
	\end{itemize}
	
	%---------- Stand van zaken ---------------------------------------------------
	
	\section{Literatuurstudie}
	%\label{sec:literatuurstudie}
	
	Evaluating storage, retention, and retrieval in disordered memory and learning \autocite{BuschkeFuld1974}
	
	Ze bespreken 2 simpele manieren om bij verschillende mensen het geheugen te quoteren en met
	elkaar te vergelijken (nl. selectief herinneren en restrictief herinneren)
	Bij selectief herinneren herhalen ze enkel de woorden die je niet genoemd hebt in de test die direct volgt na het
	opzeggen van alle woorden tot je 2 opeenvolgende testen alle woorden kan opzeggen, restrictief gaat doortot je elk woord minstens
	1x kan herinneren zodat de lijst inkrimpt (elk woord wordt na het herinneren ervan uit de lijst gehaald).
	Woorden die je kan herinneren in een sessie waar die voorgaand niet in gezegd is, komen uit je lange
	termijn geheugen. Het falen van de patiënt kunnen we linken aan het niet (snel genoeg) kunnen onthouden van woorden in het
	lange termijn geheugen. We kunnen zeggen dat een lijst gekend is als de patiënt consistent alle woorden kan herhalen, het heeft dus geen nut als
	je uit het lange termijn geheugen woorden kan onthouden wanneer je andere woorden die net gezegd zijn niet kan herhalen.
	Deze methoden zijn enkel nuttig om abnormaliteiten vast te stellen (in dit geval alcoholisme) maar vertellen weinig over de staat van het geheugen
	als je 2 patiënten met hetzelfde probleem met elkaar vergelijkt. De patiënt had bij beide methoden 6 pogingen nodig vooraleer die in staat was om alle woorden
	op te zeggen, maar was niet in staat om deze te herhalen in de volgende sessie van selectief herinneren. De beste manier om dit aan te pakken is door je visie te 
	veranderen, je mag het niet zien als 10 individuele woorden maar als een lijst waar je zelf een link moet leggen tussen verschillende woorden. Op deze manier kunnen wij
	een hypothese opstellen, "Door een link te leggen tussen de woorden zorg je niet alleen dat je in de volgende sessie een hoger aantal woorden kan herhalen maar ook 
	dat je na enkele uren meer woorden kan opsommen dan iemand die deze manier niet toepast". Deze hypothese lijkt me belangrijk voor het resultaat maar is niet behandeld in deze studie.

	
	Je mag gerust gebruik maken van subsecties in dit onderdeel.
	
	%---------- Methodologie ------------------------------------------------------
	\section{Methodologie}
	\label{sec:methodologie}
	
	Hier beschrijf je hoe je van plan bent het onderzoek te voeren. Welke onderzoekstechniek ga je toepassen om elk van je onderzoeksvragen te beantwoorden? Gebruik je hiervoor experimenten, vragenlijsten, simulaties? Je beschrijft ook al welke tools je denkt hiervoor te gebruiken of te ontwikkelen.
	
	%---------- Verwachte resultaten ----------------------------------------------
	\section{Verwachte resultaten}
	\label{sec:verwachte_resultaten}
	
	Hier beschrijf je welke resultaten je verwacht. Als je metingen en simulaties uitvoert, kan je hier al mock-ups maken van de grafieken samen met de verwachte conclusies. Benoem zeker al je assen en de stukken van de grafiek die je gaat gebruiken. Dit zorgt ervoor dat je concreet weet hoe je je data gaat moeten structureren.
	
	%---------- Verwachte conclusies ----------------------------------------------
	\section{Verwachte conclusies}
	\label{sec:verwachte_conclusies}
	
	Hier beschrijf je wat je verwacht uit je onderzoek, met de motivatie waarom. Het is \textbf{niet} erg indien uit je onderzoek andere resultaten en conclusies vloeien dan dat je hier beschrijft: het is dan juist interessant om te onderzoeken waarom jouw hypothesen niet overeenkomen met de resultaten.
	
	
	%------------------------------------------------------------------------------
	% Referentielijst
	%------------------------------------------------------------------------------
	% TODO: de gerefereerde werken moeten in BibTeX-bestand ``voorstel.bib''
	% voorkomen. Gebruik JabRef om je bibliografie bij te houden en vergeet niet
	% om compatibiliteit met Biber/BibLaTeX aan te zetten (File > Switch to
	% BibLaTeX mode)
	
	\phantomsection
	\printbibliography[heading=bibintoc]
	
\end{document}